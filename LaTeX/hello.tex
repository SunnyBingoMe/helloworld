\documentclass[a4paper,11pt, twoside]{article}

\usepackage{verbatim}
%\usepackage{algorithmicx}
%\usepackage{IEEEtrantools}

% pdftex setting
\usepackage[pdftex]{color,graphicx}
\usepackage[pdftex]{hyperref}
\hypersetup{pdfauthor=vvoody}
\hypersetup{pdftitle={hello, LaTeX!},
            colorlinks=true,
            unicode=true,
            linkcolor=blue,
            citecolor=green,
            filecolor=magenta,
            urlcolor=magenta}

% define the title
\author{vvoody}
\title{Hello, World!}

%\pagestyle{headings}

\begin{document}

% generates the title

\maketitle

% insert the table of contents

\tableofcontents

\newpage
\section{Some Interesting Words}

Well, and here begins my lovely article.

\section{test}

hello, what? \LaTeX
\label{sec:sec1}

\subsection{test-subsection}

\TeX{} hello

\subsubsection[not-too-long]{test-subsubsec-looooooooooooooooooooooooooooooooooooooooooooog}

eeeeeeeeeee

\subsubsection*{test-subsubsec-no}

OOOOOOOOOOOOOOOOOOOOO

\paragraph{test-para}

fejifej,fe fejie ,fejiaj iefj aijfe ,
fejapfhgeg8aj,feoajgaj gajga g,feaij iajge ij hellfoe,feji,
feaijg,aeg aegjai ,gmaeigioj,geagoijej

\subparagraph{subpara}

fefjajej,fe foajio jfa,feajgioa gj

\newpage
\section{Good Bye World.\protect\footnote{blablabla}}

% Example 1
\ldots when Einstein introduced his formula
\begin{equation}
e = m \cdot c^2 \; ,
\end{equation}
which is at the same time the most widely known
and the least well understood physical formula.

% Example 2
\ldots from which follows Kirchhoff’s current law:
\begin{equation}
\sum_{k=1}^{n} I_k = 0 \; .
\end{equation}

Kirchhoff’s voltage law can be derived \ldots

% Example 3
\ldots which has several advantages.

\begin{equation}
I_D = I_F - I_R
\end{equation}
is the core of a very different transistor model. \ldots
\\
\\
daughter-in-law, X-rated\\
pages 13--67\\
yes---or no? \\
$0$, $1$ and $-1$

http://www.rich.edu/\~{}bush \\
http://www.clever.edu/$\sim$demo

Mr. Smith was happy to see her\\
cf. Fig. 5\\
I like BASIC\@. What about you?\footnote{This is a footnote.}

A reference to this subsection looks like:
‘‘see section \ref{sec:sec1} on
page \pageref{sec:sec1}.’’

\newpage

\section{some formats}

Not shelfful\\
but shelf\mbox{}ful
\\
H\^otel, na\"\i ve, \'el\`eve,\\
sm\o rrebr\o d, !`Se\ norita!,\\
Sch\"onbrunner Schlo\ss{}
Stra\ss e

\verbatiminput{assignment1.pl}

\newpage

\part{part-1}

part1111111111111111111111111111111111111111111111111111111111111

\begin{figure}[!h]
\centering
\includegraphics[angle=45]{8z1.jpg}
\caption{8z monkey}
\end{figure}

The \href{http://www.ctan.org}{CTAN} website.

\url{http://www.tex.ac.uk/cgi-bin/texfaq2html?label=ifpdf}

\part{part-2}

partttttttttttttttttttttttttttttwwwwwwwwwwwwwwwwwwwwwoooooooooooooooo

The \verb|\ldots| command \ldots, the \verb*|  \LaTeX  | command \LaTeX.
\begin{verbatim}
10 PRINT ”HELLO WORLD ”;
20 GOTO 10
\end{verbatim}

\begin{verbatim*}
the starred version of
the
verbatim
environment emphasizes
the spaces
in the text
\end{verbatim*}


\begin{table}[h]
\caption{table 0}
\begin{center}
\begin{tabular}{|r|l|}
\hline
7C0 & hexadecimal \\
3700 & octal \\
\cline{2-2}
11111000000 & binary \\
\hline
\hline
1984 & decimal \\
\hline
\end{tabular}
\end{center}
\end{table}

table 1:
\begin{tabular}{|p{4.7cm}|}
\hline
Welcome to Boxy’s paragraph.
We sincerely hope you’ll
all enjoy the show.\\
\hline
\end{tabular}

table 2:
\begin{tabular}{c r @{.} l}
Pi expression
&
\multicolumn{2}{c}{Value} \\
\hline
$\pi$
& 3&1416 \\
$\pi^{\pi}$
& 36&46
\\
$(\pi^{\pi})^{\pi}$ & 80662&7 \\
\end{tabular}

table 3:
\begin{tabular}{|c|c|}
\hline
\multicolumn{2}{|c|}{Ene} \\
\hline
Mene & Muh! \\
\hline
\end{tabular}

\begin{tabular}{@{} l @{}}
\hline
no leading space\\
\hline
\end{tabular}


\begin{table}[!htp]
\begin{center}
\begin{tabular}{l}
\hline
leading space left and right\\
\hline
\end{tabular}
\caption{leading space}
\end{center}
\end{table}

Figure \ref{white} is an example of Pop-Art.
\begin{figure}[!htp]
\makebox[\textwidth]{\framebox[5cm]{\rule{0pt}{5cm}}}
\caption{Five by Five in Centimetres.\label{white}}
\end{figure}

compiled at \today.

\newpage

\appendix{Appendix}

???

\end{document}
